\documentclass[doctor,english]{thesis-uestc}  % other options: bachelor, master promaster, doctor, engdoctor
% For the template typesetting, please refer to README
\title{人工智能中常识推理的有效方法}{Efficient Methods for Commonsense Reasoning in Artificial Intelligence}
\author{Feroze Wasif}{Feroze Wasif} % % Author's name
\setdate[submit]{2022年3月17日} % Thesis submission date (Optional)
\setdate[oral]{2022年4月15日}  % Oral defense date (Optional)
\setdate[confer]{2022年6月} % Degree conferral date (Optional)
\advisor{xx \chinesespace xx}{xx xx xx} % Advisor's name and title
% \coAdvisor{合作导师姓名\chinesespace 导师职称}{Co advisor English name English title} % Only for professional master's/PhD students. Adds co-advisor on title page/English title page. Comment out if not used.
\school{计算机科学与工程学院(网络空间安全学院)}{School of Computer Science and Engineering(School of Cyberspace Security)} % School information
\major{计算机科学与技术}{Computer Science and Technology} % Major information
\studentnumber{xxxxxxx} % Student id
% \ProfessionalDegreeArea{随便学学} % Only for professional degree
\ClassificationNumber{} % Classification number (assigned by the library)
\ClassifiedClass{公开} % Confidentiality level thesis/dissertation, usually is public for international students
\UDCNumber{} % UDC number (assigned by the library)
\Chairman{xxxxx} % Chairperson of defense committee

% Uncomment the following section to prevent automatic word breaking at line breaks in English text.
% \tolerance=1
% \emergencystretch=\maxdimen
% \hyphenpenalty=10000
% \hbadness=10000

\makeglossaries % Required for generating glossary/acronym lists. Comment out if not used. Note: Also comment out acronym, glossary entries, and related references if not used.
\newacronym[description=逻辑卷管理器]{lvm}{LVM}{Logical Volume Manager} % Define an acronym. Example: 
% "逻辑卷管理器" is the Chinese name,
% "lvm" is used for in-text references,
% "LVM" is the displayed acronym or symbol,
% "Logical Volume Manager" is the full English term/description.
\newglossaryentry{tree}{name={tree}, description={trees are the better humans}}  % Define a glossary entry. Example:
% "tree" is the symbol name,
% "tree" is used for in-text references,
% "trees are the better humans" is the displayed description (with automatic page number inclusion).

\begin{document}

\makecover % Cover page + Chinese and English title pages
\originalitydeclaration % Statement of originality
% \signatureofdeclaration{signature.pdf} % Used to add a scanned version of the signed originality statement (Uncomment this line if using a scanned signature and comment out the previous line)
% Chinese abstract
\begin{chineseabstract}

    \chinesekeyword{xxx,xxx,xxx} % Chinese keywords
\end{chineseabstract}
% English abstract
\begin{englishabstract}

    \englishkeyword{xxx, xxx, xxx} % English keywords
\end{englishabstract}

\thesistableofcontents % Table of contents
\thesisfigurelist % List of figures (only include if needed; otherwise, comment out)
\thesistablelist % List of tables (only include if needed; otherwise, comment out)
% \glsaddall % By default, only glossary items cited in the text are shown. Uncomment to display all defined acronyms/symbols.
\thesisglossarylist % Glossary list (only include if needed; otherwise, comment out)
\thesissymbollist % List of symbols (only include if needed; otherwise, comment out)

% Main content starts here
\chapter{Introduction}

For Superscript citation style use\citing{chen2001hao}test, Standard citation style use ~\cite{clerc2010discrete}.

Symbol reference\gls{tree}\cite{liuxf2006}. (references a glossary entry)

$\hat{H}, f(x)$, $\vec{V}$

$$\hat{H}$$

$\mathcal{C}_i$

This is the long form reference of the acronym \acrlong{lvm}, and this is the short form reference \acrshort{lvm}.

\begin{figure}[!htb]
    \includegraphics[width=0.5\linewidth]{pic/figure.pdf}
    \caption[short catption 1]{Test caption 1}
\end{figure}


\begin{figure}[!htb]
    \small
    \centering
    \begin{tabular}{@{\ }c@{\ }c}
        \includegraphics[width=0.49\textwidth]{pic/figure.pdf} & 
        \hspace{5pt}
        \includegraphics[width=0.49\textwidth]{pic/figure.pdf}     \\
        \mbox{\small (a)A randomly attempted super long title.}                                                                               & 
        \mbox{\small (b)A randomly attempted super long title.}                                                                                  \\
    \end{tabular}
    \caption{A randomly tried super long title - Total.}
    \label{fig:test}
\end{figure}

\begin{figure}[!htbp]
    \centering
    \begin{subfigure}[t]{0.35\linewidth}
        \centering
        \includegraphics[scale=0.25]{logo.pdf}
        \caption{Fig. 1}
        \label{fig:1-1}
    \end{subfigure}
    \begin{subfigure}[t]{0.35\linewidth}
        \centering
        \includegraphics[scale=0.25]{logo.pdf}
        \caption{Fig. 2}
        \label{fig:1-2}
    \end{subfigure}
    \\[6bp]
    \begin{subfigure}[t]{0.35\linewidth}
        \centering
        \includegraphics[scale=0.25]{logo.pdf}
        \caption{Fig. 3}
        \label{fig:1-3}
    \end{subfigure}
    \begin{subfigure}[t]{0.35\linewidth}
        \centering
        \includegraphics[scale=0.25]{logo.pdf}
        \caption{Fig. 4}
        \label{fig:1-4}
    \end{subfigure}
    \caption{Fig.}
    \label{fig:1}
\end{figure}


\chapter{Related work}
\chapter{Main work 1}
\chapter{Main work 2}
\chapter{Main work 3}
\chapter{Conclusion and Future work}

Algorithm template:

\begin{algorithm}[H]\label{alg:1}
    \KwData{this text}
    \KwResult{how to write algorithm with \LaTeX2e}
    initialization\;
    \While{not at end of this document}{
        read current\;
        \eIf{understand}{
            go to next section\;
            current section becomes this one\;
        }{
            go back to the beginning of current section\;
        }
    }
    \caption{How to wirte an algorithm.}
\end{algorithm}

This is the algorithm.\algref{alg:1}。

\thesisacknowledgement

This is my acknowledgment. % Directly fill in the acknowledgment content, with the same writing style as the main text.

\thesisbibliography[large]{reference} % references

\thesisappendix
\chapter{Appendix 1} % Directly fill in the appendix content, with the same writing style as the main text.

% List your achievements (list of published papers for doctor candidates):
\begin{thesistheaccomplish}
    \section{Academic papers}
    \bibitem{SGXDedup} \textbf{Ren, Yanjing} and Li, Jingwei and Yang, Zuoru and Lee, Patrick PC and Zhang, Xiaosong. Accelerating Encrypted Deduplication via SGX[C]. Proc.of USENIX ATC, 2021, 957-971. \textbf{CCF-A}
    \section{Invention patents}
    \bibitem{CN111338572B} 李经纬, 杨祚儒, \textbf{任彦璟}, 李柏晴, 张小松. 一种可调节加密重复数据删除方法:CN111338572B[P]. 2021-09-14.
\end{thesistheaccomplish}

% The following is for undergraduate students only, insert translation of the references.
% comment or uncomment if necessary for you
% \thesistranslationoriginal
% \section{Tahoe-LAFS: The Least-Authority File System}
% \insertPDFPage{} % "Used to insert a single-page PDF file, such as the original literature (this operation will cause the page numbers to be reset, please use with caution)."

% \thesistranslationchinese
% \section{Tahoe-LAFS:最小权限文件系统}

\end{document}
